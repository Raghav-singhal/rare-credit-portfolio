Local Intensity model belongs to family of Markovian intensity models of credit risk.  This model is applied using IPS framework as mentioned in \cite{carmona2009importance}.
\subsubsection{Local Intensity Model as Markov Point Process}
 Assume there are $n$ names in the credit risk portfolio. The calculation of cumulative loss $L = {L_{t};t \geq 0}$ in the credit portfolio is modeled as a Markov point process as mentioned in \cite. This Markov point process is characterized by the local intensity function $\lambda(t,L_{t})$ given by a deterministic function ${\lambda(t,i)}_{t \geq 0,i \geq 0}$ satisfying $\lambda(t,i) = 0$ for $ i \geq n$. We need to satisfy the last condition to ensure that process $L$ is stopped when $n^{th}$ level as $n$ that is the maximum number of defaults that can occur because there are $n$ names in the portfolio. We assume that $L_{0} = 0$ and $L$ is considered as a pure birth process. Thus, the probability of a jump in the infinitesimal time interval $(t,t+dt)$ is given by $\lambda(t,L_{t})dt$. \\
The infinitesimal generator $\mathcal{G}_{t}$ of the process is given by the $(n+1) \times (n+1)$ matrices: \\
\[\mathcal{G}_{t} = 
	\begin{bmatrix}
-\lambda(t,0)& \lambda(t,0) & 0 & 0 & 0 \\
0 & -\lambda(t,1)& \lambda(t,1)  & 0 & 0 \\
 & &\dots & & \\
 0 & 0 & 0 & -\lambda(t,n-1)& \lambda(t,n-1) \\
 0 & 0 & 0 & 0 & 0
\end{bmatrix}
\]
Now lets consider $p_{i}(t) = \mathbb{P}{L_{t}=i}$ and $p(t) = [p_{i}(t)]_{i=0 \dots n}$ and it satisfies Kolmogorov equation:
\begin{equation}
\left(\partial_{t} - \mathcal(G)_{t}^{*}\right)p = 0 \text{ on } (0,+\infty), p(0) = \delta_{0},
\end{equation} 
which is in fact a system of ordinary differential equations where $\mathcal{G}_{t}^{*}$ denotes the transpose of the $\mathcal{G}$ matrix as it is a real matrix. Now Kolmogorov equation can be expressed in terms of possibly infinitely many matrix multiplications. And hence the solution to the Kolmogorov equation is given by a matrix exponential  of the form \\
\begin{equation}
p(t) = exp(t\mathcal{G}^{*})\delta_{0}, \quad t \geq 0
\end{equation}
 We also set $\tilde{t_{0}} = 0 $. Given a fixed maturity $T$, say $ T = 5yr$, we define $t_{i} = min(\tilde{t_{i}},T)$ for each $i = 0, \dots, n$. Consequently, $t_{i} < T$ and $L_{t_{i}} = i$ if and only if there are at least i jumps of $L$ before the maturity $T$ on a given trajectory. 

\subsubsection{Modeling Markov Jumps}
The sample paths of the loss process $L$ are piecewise constant. Let $\tilde{t_{i}}$ denote the $i$th ordered jump time of L, for each $i = 1, \dots , n,$ (or $\tilde{t_{i}}  = + \infty$, in case there are less than $i$ jumps on a given trajectory). For the implementation of the algorithm the time steps are chosen randomly from exponential distribution according to a given formula:
\begin{equation}
\label{eqn:evolve}
\Delta t \sim \frac{1}{1-\frac{L_{t}}{n}} \times \exp(-x)
\end{equation}
\begin{equation}
\tilde{t_{i}} = t_{i-1} +  \Delta t
\end{equation}
$t_{i} = min(\tilde{t_{i}},T)$ for each $i = 0, \dots, n$ where T is maturity time. Consequently, $t_{i} < T$ and $L_{t_{i}} = i$ if and only if there are at least i jumps of $L$ before the maturity $T$ on a given trajectory. The Markov chain ${X_{i}}_{0 \leq i \leq n}$ is defined by: \\
\begin{equation}
X_{i} = (t_{i}, L_{t_{i}}), 0 \leq i \leq n.
\end{equation}
We use this notation of Markov chain through out the paper for the discussion of Local Intensity Models. 
We define the potential function for Local Intensity Model that is used in the Selection Process of IPS so as to assign more weight to portfolios with lower values. This will help in assigning more weights to rarer events so that likelihood of default increases. The potential is a function of $X_{i}$ and $\alpha$. However, in this case, $X_{i}$ is just used to determine if more weights should be put on the jumps that will increase the defaults. The potential function is defined as follows:
\begin{equation}
\omega^{\alpha}(X) = 
\begin{cases}
\exp(\alpha), & \text{if } t < T\\
1,               & \text{otherwise}
\end{cases}
\end{equation}
The values of $\alpha$ parameter will play a crucial role in determining the important samples that will help in increasing the number of defaults. The $\alpha$ that gave the best results was chosen by the given formula:
\begin{equation}
\alpha(\ell) = argmax_{\alpha} \# \{ j; 0 \leq j \leq m, L_{T}(\omega_{\alpha}^{j},\alpha) = \ell \}
\end{equation}
