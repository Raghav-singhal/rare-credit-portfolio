% arara: indent: {overwrite: yes}
\documentclass{article}

% if you need to pass options to natbib, use, e.g.:
% \PassOptionsToPackage{numbers, compress}{natbib}
% before loading nips_2016
%
% to avoid loading the natbib package, add option nonatbib:
% \usepackage[nonatbib]{nips_2016}

\usepackage[final]{nips_2016}

% to compile a camera-ready version, add the [final] option, e.g.:
% \usepackage[final]{nips_2016}

\usepackage[utf8]{inputenc} % allow utf-8 input
\usepackage[T1]{fontenc}    % use 8-bit T1 fonts
\usepackage{hyperref}       % hyperlinks
\usepackage{url}            % simple URL typesetting
\usepackage{booktabs}       % professional-quality tables
\usepackage{amsfonts}       % blackboard math symbols
\usepackage{nicefrac}       % compact symbols for 1/2, etc.
\usepackage{microtype}      % microtypography
\usepackage{physics}		 % Physics package for derivatives
\usepackage{graphicx,float,caption,subcaption,tikz}
\usepackage[nomarkers,figuresonly]{endfloat}
\usepackage{listings}
\usepackage{color} %red, green, blue, yellow, cyan, magenta, black, white
\usepackage{animate}
\usepackage{multirow}
\usepackage{pgffor}
\usepackage{tabularx}

\usepackage{amsmath}
\usepackage{latexsym}
\usepackage{amssymb}
\usepackage{mathtools}
\usepackage{bm}
\usepackage{array}
\usepackage{color} %red, green, blue, yellow, cyan, magenta, black, white


\definecolor{mygreen}{RGB}{28,172,0} % color values Red, Green, Blue
\definecolor{mylilas}{RGB}{170,55,241}
\captionsetup[table]{skip=10pt}

\DeclareMathOperator*{\argmax}{\arg\max}
\DeclareMathOperator*{\argmin}{\arg\min}
\DeclareMathOperator*{\E}{\mathbb{E}}

\bibliographystyle{abbrvnat}
\graphicspath{ {images/}}

\lstset{language=Python,%
    basicstyle=\footnotesize\ttfamily,
    breaklines=true,%
    morekeywords={matlab2tikz},
    keywordstyle=\color{blue},%
    morekeywords=[2]{1}, keywordstyle=[2]{\color{black}},
    identifierstyle=\color{black},%
    stringstyle=\color{mylilas},
    commentstyle=\color{mygreen},%
    showstringspaces=false,%without this there will be a symbol in the places where there is a space
    numbers=left,%
    numberstyle={\tiny \color{black}},% size of the numbers
    numbersep=9pt, % this defines how far the numbers are from the text
    %emph=[1]{for,end,break},emphstyle=[1]\color{red}, %some words to emphasise
    %emph=[2]{word1,word2}, emphstyle=[2]{style},    
}

\newcounter{codenum}
\newcounter{imgnum}
\makeatletter
\newtoks\@tabtoks
\newcommand\addtabtoks[1]{\global\@tabtoks\expandafter{\the\@tabtoks#1}}
\newcommand\eaddtabtoks[1]{\edef\mytmp{#1}\expandafter\addtabtoks\expandafter{\mytmp}}
\newcommand*\resettabtoks{\global\@tabtoks{}}
\newcommand*\printtabtoks{\the\@tabtoks}
\makeatother

\newcommand*\diff{\mathop{}\!\mathrm{d}}
\newcommand*\Diff[1]{\mathop{}\!\mathrm{d^#1}}
\newcommand{\me}{\mathrm{e}}

\renewcommand{\efloatseparator}{\mbox{}}

\title{Rare Event Simulation using Interacting Particle Systems for Rare Credit Portfolio Losses}

% The \author macro works with any number of authors. There are two
% commands used to separate the names and addresses of multiple
% authors: \And and \AND.
%
% Using \And between authors leaves it to LaTeX to determine where to
% break the lines. Using \AND forces a line break at that point. So,
% if LaTeX puts 3 of 4 authors names on the first line, and the last
% on the second line, try using \AND instead of \And before the third
% author name.

\author{
  Abhishek Shah \\
  \texttt{ans556@nyu.edu}\\
  Computer Science,  NYU
  \And
  Prithvi Krishna Gattamaneni\\
  \texttt{pkg238@nyu.edu}\\
  Computer Science,  NYU
  \AND
  Raghav Singhal\\
  \texttt{rs4070@nyu.edu}\\
  Mathematics, NYU
  \And
  Srivas Venkatesh \\
  \texttt{sv1358@nyu.edu} \\
  Computer Science,  NYU
}

\begin{document}

\maketitle

\begin{abstract}
	With the recent explosion of the credit market, it is imperative to understand
	the risk profiles of these large credit portfolios. This helps segment and
	structure the portfolio based on the risk and price it appropriately. However
	this is a very high dimensional problem and suffers from the typical issues
	of such problems. Moreover the joint probability of multiple defaults
	occurring is very low and hence estimating such probabilities is harder too.
	In this project we explore Interacting Particle Systems methodology to sample 
	these rare events and estimate the default probabilities from these simulations.
	We also explore the use of a few different models for the credit portfolios.
\end{abstract}
\section{Motivation}
\begin{frame}

Given a Markov chain $S=(S_n)_{0 \leq n \leq T}$. At each n, we have $d$ correlated assets, $S_n = (S_n^1,...,S_n^d) \in \mathit{E}$. We wish to understand the nature of probabilities of rare events which are of the form $L(T) > K$, where $L(T)$ is the number of defaults at the maturity time $T$, can be expressed as below.
$$
\{L(T) \geq K\} = \{V_T(S_T)\geq K\} = \{V_T(S_0,...,S_T)\geq K\}
$$
$V(T)$ can be thought of as a real positive function that is a risk measure of these rare events.\\

$$\mathbb{P}(V_T(S_T) \geq K) = \mathbb{E}\left( \mathbf{1}_{\{V_T(S_T) \geq K\}}e^{\lambda V_T(S_T)}e^{-\lambda V_T(S_T)} \right)$$
can be rewritten as
$$\mathbb{E}^{(\lambda)} \left(  \mathbf{1}_{\{V_T(S_T) \geq K\}} e^{-\lambda V_T(S_T)} \right) \mathbb{E} \left(e^{\lambda V_T(S_T)}\right) = 
\mathbb{E}^{(\lambda)}(f_T(S_T)) \mathbb{E}(e^{\lambda V_T(S_T)})  $$
where $f_t(S_T) := \mathbf{1}_{\{V_T(S_T) \geq K\}}e^{-\lambda V_T(S_T)} $. 
\end{frame}

\begin{frame}
With the convention that $V_0 = 0$, we get the following decomposition
$$e^{\lambda V_T(S_T)} \equiv \prod_{p=1}^{T} e^{\lambda (V_p(S_p) - V_{p-1}(S_{p-1}))}$$

By using the notation $$\mathcal{X}_k = (S_k, S_{k+1})$$ for $0 \leq k < T$, the above produce can be rewritten as

$$\prod_{p=1}^{T} G_{p-1}(\mathcal{X}_{p-1})$$ where
$$G_{p-1}(\mathcal{X}_{p-1}) := e^{\lambda (V_p(S_p) - V_{p-1}(S_{p-1}))}$$

Using the notation that $F_T(\mathcal{X}_T) = f_T(S_T)$ we get
\begin{equation}
	\mathbb{E}^{(\lambda)}(f_T(S_T)) = \frac{\mathbb{E}(F_T(\mathcal{X}_T)\prod_{p=1}^{T}G_p(\mathcal{X}_p))}{\mathbb{E}(\prod_{p=1}^{T}G_p(\mathcal{X}_p))} := \eta_T(F_T)
\end{equation}
\end{frame}
\section{Theory}
\subsection{Interacting Particle Systems}
Particle methods depend upon the existence of a background Markov chain which is denoted by $X = {X_n}_{n\geq 0}$. This chain is not assumed to be time homogeneous. The random element $X_n$ takes values in some measurable state space $(E_n, \varepsilon_n)$ that can change with n. $K_n(x_{n-1}, dx_n)$ is the Markov transition kernels. $B_b(E)$ is the space of bounded measurable functions of the measurable space $(E,\varepsilon)$

\subsubsection{Feynman-Kac Path Expectations}
Let $Y_n$ be the history of $X_n$.
$$Y_n := (X_0,...,X_n) \in F_n := (E_0 \cross ... \cross E_n), n \geq 0$$

${Y_n}_{n\geq 0}$ is itself a Markov chain and we denote by $M_n(y_{n-1},dy_n)$ its transition kernel. For each $n \geq 0$, we choose a multiplicative potential function $G_n$ defined on $F_n$ and we define the Feynman-Kac expectations by

\begin{equation}
\gamma_n(f_n) = \mathbb{E} \left[ f(Y_n) \prod_{1\leq p<n} G_k(Y_p)\right]
\end{equation}

We define $\eta_n(.)$ the corresponding normalized measure defined as
\begin{equation}
\eta_n(f_n) = \frac{\mathbb{E}\left[ f_n(Y_n)\prod_{1\leq k<n}G_k(Y_k) \right]}{\mathbb{E}\left[\prod_{1\leq k<n}G_k(Y_k) \right]} = \gamma_n(f_n)/\gamma_n(1).
\end{equation}

We also observe that
$$\gamma_{n+1}(1) = \gamma_n(G_n) = \eta_n(G_n)\gamma_n(1) = \prod_{p=1}^{n}\eta_p(G_p)$$ 

Hence given any bounded measurable function $f_n$ on $F_n$ we have
$$\gamma_n(f_n) = \eta_n(f_n) \prod_{1\leq p<n}\eta_p(G_p)$$

Using the notation $G_p^- = 1/G_p$ and the above definitons of $\gamma_n$ and $\eta_n$ we see that

\begin{eqnarray}
\mathbb{E}[f_n(Y_n)] & = & \mathbb{E}\left[\ f_n(Y_n) \prod_{1\leq p < n} G_p^-(Y_p) \cross \prod_{1\leq p < n} G_p(Y_p) \right]\\
& = & \gamma_n\left(  f_n \prod_{1\leq p < n} G_p^- \right)\\
& = & \eta_n\left(  f_n \prod_{1\leq p < n} G_p^- \right) \prod_{1\leq p < n} \eta_p(G_p)
\end{eqnarray}

We can also check by inspection that the measures $(\eta_n)_{n\geq 1}$ satisfy the nonlinear recursive equation.
$$\eta_n = \varPhi_n(\eta_{n-1}) := \int_{F_{n-1}} \eta_{n-1}(dy_{n-1}) \frac{G_{n-1}(y_{n-1})}{\eta_{n-1}(G_{n-1})} M_n(y_{n-1},.)$$
starting from $\eta_1 = M_1(x_0,.)$. This dynamical equation on the space of measures is known as Stettners equation in filtering theory. We state it to justify the selection/mutation decomposition of each step of the particle algorithm introduced below.


\subsubsection{IPS Interpretation and Monte Carlo Algorithm}
We introduce a natural interacting path-particle system. For a given integer M, using the transformatios $\varPhi_n$, we construct a Markov chain $\{\xi_n\}_{n\geq 0}$ whose state $\xi_n = (\xi_n^j)_{1\leq j \leq M}$ at time n can be interpreted as a set of M Monte Carlo samples of path-particles 
$$\xi_n^j = (\xi_{0,n}^j, \xi_{1,n}^j,..., \xi_{n,n}^j) \in F_n = (E_0 \cross ... \cross E_n)$$
The transition mechanism of this Markov chain can be described as follows. We start with an initial configuration $\xi_1 = (\xi_1^j)_{1\leq j \leq M}$ that consists of M independent and identically distributed random variables with distribution,

$$\eta_1(d(y_0,y_1)) = M_1(x_0,d(y_0,y_1)) = \delta_{x_0}(dy_0)K_1(y_0, dy_1) $$
i.e $\xi_1^j := (\xi_{0,1}^j,\xi_{1,1}^j) = (x_0, \xi_{1,1}^j) \in F_1 = (E_0 \cross E_1)$ where the $\xi_{1,1}^j$ are drawn independently of each other from the distribution $K_1(x_0,.)$. Then the one-step transition taking $\xi_{n-1} \in F_{n-1}^M$ into $\xi \in F_n^M$ is given by a random draw from the distribution

\begin{equation}
\mathbb{P}\big( \xi_n \in d(y_n^1,...,y_n^M)|\xi_{n-1}  \big) = \prod_{j=1}^{M}\varPhi(m(\xi_{n-1}))(dy_n^j)
\end{equation}

where the notation $m(\xi_{n-1})$ is used for the empirical distribution of the $\xi_{n-1}^j$, i.e
$$m(\xi_{n-1}) := \frac{1}{M}\sum_{j=1}^{M}\delta_{\xi_{n-1}^{j}}$$

From the definition of $\varPhi_n$, we see that the above is the superposition of a section procedure followed by a mutation given by the transition of the original Markov chain. More precisely,
$$\xi_{n-1} \in F_{n-1}^{M}  \xrightarrow{\text{selection}} \hat{\xi}_{n-1} \in F_{n-1}^M \xrightarrow{\text{mutation}} \xi_n \in F_n^M$$

where the selection stage is performed by choosing randomly and independently M path-particles
$$\hat{\xi}_{n-1}^j = (\hat{\xi}_{0,n-1}^j, \hat{\xi}_{1,n-1}^j,...,\hat{\xi}_{n-1,n-1}^j) \in F_{n-1}$$
according to the Boltzmann-Gibbs measure
\begin{equation}
\sum_{j=1}^{M} \frac{G_{n-1}(\xi_{0,n-1}^j,...,\xi_{n-1,n-1}^j)}{\sum_{k=1}^{M}G_{n-1}(\xi_{0,n-1}^k,...,\xi_{n-1,n-1}^k)} \delta_{(\xi_{0,n-1}^j,...,\xi_{n-1,n-1}^j)}
\end{equation}
and for the mutation stage, each selected part-particle $\hat{xi}_{n-1}^j$ is extended as follows.
\begin{eqnarray*}
\xi_n^j & = & (\hat{\xi}_{n-1}^j, \xi_{n,n}^j)\\
& = & ((\hat{\xi}_{0,n-1}^j,..., \hat{\xi}_{n-1,n-1}^j), \xi_{n,n}^j) \in F_n = F_{n-1} \cross E_n
\end{eqnarray*}
where $\xi_{n,n}^j$ is a random variable with distribution $K_n(\hat{\xi}_{n-1,n-1},.)$. In other words, the transition step is a mere extension of the path particle with an element drawn at random using the transition kernel $K_n$ of the original Markov chain. All of the mutations are performed independently. But most importantly all of these mutations are happening with the original transition distribution of the chain. This is a contrast with importance sampling where the Monte Carlo transitions are from twisted transition distributions obtained from a Girsanov-like change of measure. So from a practical approach, a black-box providing random samples from the original chain transition distribution is enough for the implementation of the IPS algorithm.\\

Another result states that for each fixed time n, the empirical historical path measure
$$\eta_n^M := m(\xi_n) = \frac{1}{M} \sum_{j=1}^{M} \delta_{\xi_{0,n}^j,\xi_{1,n}^j,...,\xi_{n,n}^j}$$ converges in distribution as $M \rightarrow \infty$, toward the normalized Feynman-Kac measure $\eta_n$. Moreover there are several propagation of chaos estimates that ensure that $(\xi_{0,n}^j, \xi_{1,n}^j,...,\xi_{n,n}^j)$ are asymptotically independent and identically distributed with distribution $\eta_n$. This justifies for each measurable function $\tilde{f}_n$ on $F_n$ the choice of
\begin{equation}
\gamma_n^M(\tilde{f}_n) = \eta_n^M(\tilde{f}_n) \prod_{1\leq p <n}\eta_p^M(G_p)
\end{equation}
for a particle approximation of the expectation $\gamma_n(\tilde{f}_n)$


\subsection{Markovian Intensity Models}
\subsection{Merton's Model with Stohastic Volatility}
This is a model for structural credit risk based on Merton's model with a
stochastic volatility term. We then apply IPS system on this model to estimate
the rare default probabilities as described in \cite{CarmonaIPS}. The details of
the model and application of IPS to this model is mentioned below.
\subsubsection{Credit Portfolio Model}
Given a portfolio of credit instruments related to $N$ firms, where each 
underlying asset evolves according to the following SDE:

\begin{equation}
	\label{eq:merton_asset_sde}
	dS_{i}(t) = rS_{i}(t)dt + \sigma_{i}\sigma(t)S_{i}(t)dW_{i}(t)
\end{equation}

where $r$ is the risk-free interest rate, $\sigma_{i}$
is a non-random volatility factor, and the correlation structure of the driving 
Wiener processes $W_{i}$ is given by:
\begin{equation}
	d \langle W_{i}, W_{j} \rangle_{ t} = \rho_{ij} dt
\end{equation}

and $\sigma(t)$ evolves according to another stochastic differential equation:
\begin{equation}
	\label{eq:merton_volatility_sde}
	d\sigma(t) = \kappa(\bar{\sigma} - \sigma(t)) dt + \gamma \sqrt{\sigma(t)} dW(t)
\end{equation}
where $\kappa,\bar{\sigma},\gamma$ are constants and the Wiener Process 
satisfies $\forall i = 1,2.....,N$:
\begin{equation}
	d \langle W_{i}, W \rangle _{t} = \rho_{\sigma }dt
\end{equation}

Now, for each asset we take a deterministic barrier, $B_{i}(t)$, or in other 
words a threshold , so that if the asset price falls under that barrier price 
at any time, the firm defaults. We then define a stopping time $\tau_{i}$:

\begin{equation}
	\tau_i = inf\left\lbrace t \geq 0 : S_{i}(t) \leq B_{i}(t) \right\rbrace
\end{equation}

We now define the Portfolio Loss Function $L(t)$ as the number of defaults till 
a given time $t$:
\begin{equation}
	L(t) = \sum_{i =0}^{n} \mathbf{1}_{\lbrace\tau_{i} \leq t \rbrace}
\end{equation}

Since the spreads of CDO tranches are derived from the knowledge of a finite
number of expectations of the form:
\begin{equation}
	\mathbf{E}[(L(T)-K)^{+}]
\end{equation}
where $T$ is the coupon payment date and $K$ is an acceptable number of defaults, 
beyond which we start accumulating losses. So to evaluate such expectation, we 
estimate the probabilities of default. For that purpose, we evaluate 
$\forall k = 0,1,.....,N$
\begin{equation}
	\mathbb{P}(L(T)=k) = \mathbf{p}_{k}(T)
\end{equation}

\subsubsection{Discretization of the Model}
\label{subsubsec:merton_discretization}
For the implementation of our algorithm and for computational efficiency we 
select two time step, $\Delta t = \frac{1}{20}$ which is used to perform the 
selection step, and $\delta t = 10^{-3}$ which will be used in the Euler Step.

The Markov Chain that we simulate is given as ( Note that $X_n$ is $2N + 1$ dimensional):
\begin{equation}
	\label{eq:merton_markov_chain}
	X_{n} = \left( \sigma \left( n \Delta t \right), \left( S_i \left( n \Delta
	t\right) \right)_{1 \leq i \leq N} , \min_{0 \leq m \leq n} \left( \left( S_i
	\left( m \Delta t \right) \right) \right) \right)
\end{equation}

A remarkable and distinctive feature of the Interacting Particle System approach 
is that the evolution dynamics of the underlying process is preserved, that is 
the Markov Chain $X_n$ follows the same evolution dynamics as the continuous Model.

We also take a constant Barrier, $B_i =36$, and define the stopping time $\tau_i$ as :

\begin{equation}
	\tau_i = \min \{ n \geq 0 : S_i(n \Delta t) \leq B_i \}
\end{equation}

Now, we define the potential function such that we assign more weight to portfolios 
with lower values, that is we assign more weight to rare events so the likelihood 
of defaults increases. The potential is a function of $X_p$ and another 
parameter $\alpha > 0$:
\begin{equation}
	\label{eq:merton_potential}
	G_{p}(Y_{p}) = \exp[-\alpha (V(X_p) - V(X_{p-1}))]
\end{equation}

where $V(X_p) = \sum_{i=1}^N \log (min_{0\leq m  \leq p}S_{i}(m \Delta t))$ 
so the potential can be written as:
\begin{equation}
	G_{p}(Y_{p})= \exp\left[ -\alpha \sum_{i=1}^{N}\log\frac{min_{0\leq m  \leq p}
		S_{i}(m \Delta t)}{min_{0\leq m  \leq p - 1}S_{i}(m \Delta t)}\right]
\end{equation}

where $Y_{p} =(X_{0},X_{1},...,X_{p})$ but we only need the last two values 
not the earlier values. Notice that different values of $\alpha$ will give 
different Loss Distributions $\mathbb{P}(L(T) = k)$ for all $k$, because in 
the selection step those portfolios with lower number of defaults are assigned 
a lower weight leading to different number of defaults in each portfolio. The 
choice of the potential function and the parameter $\alpha$ lead to enough 
number of sample paths with large number of defaults even if the number of 
samples is significantly lower than what would be required by a plain Monte 
Carlo simpler. We however follow an idea mentioned in 
\cite{carmona2009importance}, where the best $\alpha$ is selected for each $k$.

\subsubsection{Single Asset Constant Volatility Analysis}
\label{subsubsec:single_asset}
The simplest case of the above model is a case of single asset and no stochastic
volatility. As we know the exact solution for Geometric Brownian Motion, we can 
obtain the hitting time distribution analytically. This can then be used to
compare against estimated values. The solution for the Geometric Brownian Motion
is as follows.
\begin{equation}
	S_t = S_0 \exp \left( \left( r - \frac{\sigma^2}{2} \right) t  + \sigma W_t\right)
\end{equation}
which implies that
\begin{equation}
	\begin{split}
		\mathbb{P}[\tau_B \leq T] = \mathbb{P}[\min_{t\leq T} S_t \leq B] 
		&= \mathbb{P}[\min_{t\leq T} (r - \frac{\sigma^2}{2}) t  + \sigma W_t \leq 
		\log \frac{B}{S_0}] \\
		&= \mathbb{P}[\min_{t\leq T} (r - \frac{\sigma^2}{2}) t  + \sigma \sqrt{t} 
		N(0,1) \leq \log \frac{B}{S_0}]
	\end{split}
\end{equation}

Now, using Girsonov's Theorem, we can get an analytic expression for the 
distribution above:
\begin{equation}
	\mathbb{P}[\tau_B \leq T] = 1 - \left( \Phi(d^+) - \left( \frac{S_0}{B}
	\right)^p \Phi(d^-) \right)
\end{equation}
where :
\begin{equation*}
	d^+ = \frac{ \log \frac{S_0}{B} + (r - \frac{\sigma^2}{2}) t }{\sigma \sqrt{T}}
\end{equation*}
\begin{equation*}
	d^- = \frac{ - \log \frac{S_0}{B} + (r - \frac{\sigma^2}{2}) t }{\sigma \sqrt{T}}
\end{equation*}
\begin{equation*}
	p =  1 - \frac{2r}{\sigma^2}
\end{equation*}
and $\Phi$ is the cumulative distribution function for a standard normal.


\section{Algorithm}
\subsection{IPS algorithm for Markovian Intensity Models}
\subsection{IPS Algorithm for Modified Merton's Model}
\subsection{IPS Algorithm for Modified Merton's Model}
We let $\Delta t = \frac{T}{n}$ and divide the Time interval $[0,T]$ in to equal 
intervals. We denote the chain $X_p = \tilde{X}_{\frac{pT}{n}}$, where $\tilde{X}$ 
evolves according to the continuous time dynamics,and denote the whole history of 
the chain as $Y_{p} =(X_{0},X_{1},...,X_{p})$. We use the potential function 
defined in the equation (\ref{eq:merton_potential}). And we select a smaller 
time step to calculate the Euler Step, and for our experiment we have chosen 
$\delta t = 10^{-3}$.

Also due to the form of the potential function, we do not have to track the 
entire history of the particle, only its current value $X_p$ and that of its 
"parent" $X_{p-1}$, which is denoted as $\hat{W}_p$ in the following description.

\subsubsection{Initialization}
We take $M$ particles, where each particle represents a complete portfolio, 
with identical initial values. So $\forall j \in \{,1,..,M\},$
\begin{equation}
	\hat{X}_0^{j} = \left( \sigma(0), \left( S_1(0), \cdots, S_N(0) \right)_{1 \leq i \leq N} , 
	\left( S_1(0), \cdots, S_N(0) \right) \right)
\end{equation}

And we define the initial parent $\hat{W}_0^{j}=\hat{X}_0^{j}$.

\subsubsection{Selection Stage}
Suppose at time $p$ we have a set of $M$ particles, $(\hat{W}_p^{j},\hat{X}_p^{j})$
, with $1 \leq j \leq M$. We then compute a normalization constant $\hat{\eta}_{p}^{M}$ as:
\begin{equation}
	\hat{\eta}_{p}^{M} = \frac{1}{M} \sum_{j=1}^{M} \exp \left[ \alpha \left( 
	V(\hat{X}_{p}^{(j)}) \right) - V(\hat{W}^{(j)}_{p}) \right]
\end{equation}

Then we choose $M$ independent samples using the following distribution:

\begin{equation}
	\eta_{p}^{M} (dW,dX) = \frac{1}{M \hat{\eta}_{p}^{M}} \sum_{j=1}^{M} 
	\exp \left[ \alpha \left( V(\hat{X}_{p}^{(j)}) \right) - V(\hat{W}^{(j)}_{p}) 
	\right] \times \delta_{(\hat{W}_p^{j},\hat{X}_p^{j})} (dW,dX)
\end{equation}

The particles selected are then denoted as $(\breve{W}_p^{j},\breve{X}_p^{j})$.

\subsubsection{Mutation Stage}
This stage sets the IPS apart from other importance sampling methods, as we use 
the exact dynamics of the model to sample points. We chose the Euler-Maruyama 
method to solve for the Asset Prices and the Stochastic Volatility with the 
time step $\delta t$ mentioned above.

For each particle $(\breve{W}_p^{j},\breve{X}_p^{j})$, we evolve it using the 
Euler-Maruyama scheme from $t_p$ to $t_{p+1}$, so $\breve{X}_p^{j}$ becomes $\hat{X}_{p+1}^{j}$.
Note that, each particle, that is a portfolio, is evolved independently.

\subsubsection{Termination Stage}
At the Maturity Time, we compute the number of losses in each particle, 
that is a portfolio, for all $M$ particles by computing the function $f_n$ 
defined as follows:

\begin{equation}
	f(X^{(j)}_n) = \sum_{i=1}^{N}\mathbf{1}_{\lbrace X^{(j)}_{n}(N + 1 + i)\leq B_{i}\rbrace}
\end{equation}

where the last $N$ components of $X_n$ are the minimums of the asset values. 
The estimates $\hat{p}_k^M(T)$ for the number of defaults, $\mathbf{p}_k(T) = \mathbb{P}(L(T)=k)$
, is defined as:
\begin{equation}
	\hat{p}_{k}^{M}(T) = \left[ \frac{1}{M} \sum_{j=1}^{M} \mathbf{1}_{\lbrace 
			f(\hat{X}_{n}^{(j)}) = k \rbrace } \exp \left[ \alpha \left( V(\hat{W}^{(j)}) -
		V(\hat{X_{0}}) \right) \right] \right] \times \left[ \prod_{p=0}^{n-1} 
			\hat{\eta}_{p}^{M} \right]
		\end{equation}
		As explained in the theory section, the above estimator is an unbiased estimator.


\section{Results}
\begin{figure}
    \centering
    \includegraphics[width = \textwidth]{IPS_all_years}
    \caption{Plot of default probabilities for each level and maturity times
    based on IPS}
    \label{fig:IPS_all}
\end{figure}

\begin{figure}
    \centering
    \includegraphics[width = \textwidth]{IPS_M_comparison}
    \caption{Plot comparing default probabilities for maturity $T=1$ with
        different number of portfolios}
    \label{fig:IPS_comparison}
\end{figure}

\begin{figure}
    \centering
    \includegraphics[width = \textwidth]{IPS_surface}
    \caption{Plot of number of defaults simulated for $\alpha$, $k$ combination}
    \label{fig:IPS_surface}
\end{figure}



\clearpage
\nocite{code}
\bibliography{references}
\end{document}
