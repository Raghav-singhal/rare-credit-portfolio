For this particular setup, the 3 experiments were performed. The parameters were set to the given values:
\begin{center}
	\begin{tabular}{|c|c|}
		\hline
		Parameter      & Value                              \\
		\hline
		\hline
		n & 125 \\
		\hline
		$\alpha$ & 0.4 - 4.0 \\
		\hline
		$\alpha$-step & 0.4 \\
		\hline
		M & 5000,20000,100000 \\
		\hline
	\end{tabular}
\end{center}

In figure (a) of \ref{fig:LIM_1m}, \ref{fig:LIM_20k} and \ref{fig:LIM_5k}, the relationship between $\alpha$ values and the number of defaults is shown. It can be observed that there is large difference in the number of defaults with the increase in $\alpha$. Infact, with the change of $\alpha$, there is significant change in the number of defaults. Hence, choice of $\alpha$ plays a crucial role in simulating rare defaults. For example, the results of m=5000, 1153 portfolios were able to simulate 40 defaults. 

In figure (b) of \ref{fig:LIM_1m}, \ref{fig:LIM_20k} and \ref{fig:LIM_5k}, the relationship between number of defaults and the log probabilities are shown. Also, changing the values of m drastically, doesn't change the log probabilties which explains the efficiency of the algorithm. However, it can be also observed that for different number of samples, the probabilities obtained for the higher number of defaults are significantly low. The number of defaults improve from normal monte carlo case when the value of $\alpha=0$. 