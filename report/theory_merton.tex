This is a model for structural credit risk based on Merton's model with a
stochastic volatility term. We then apply IPS system on this model to estimate
the rare default probabilities as described in \cite{CarmonaIPS}. The details of
the model and application of IPS to this model is mentioned below.
\subsubsection{Credit Portfolio Model}
Given a portfolio of credit instruments related to $N$ firms, where each 
underlying asset evolves according to the following SDE:

\begin{equation}
	\label{eq:merton_asset_sde}
	dS_{i}(t) = rS_{i}(t)dt + \sigma_{i}\sigma(t)S_{i}(t)dW_{i}(t)
\end{equation}

where $r$ is the risk-free interest rate, $\sigma_{i}$
is a non-random volatility factor, and the correlation structure of the driving 
Wiener processes $W_{i}$ is given by:
\begin{equation}
	d \langle W_{i}, W_{j} \rangle_{ t} = \rho_{ij} dt
\end{equation}

and $\sigma(t)$ evolves according to another stochastic differential equation:
\begin{equation}
	\label{eq:merton_volatility_sde}
	d\sigma(t) = \kappa(\bar{\sigma} - \sigma(t)) dt + \gamma \sqrt{\sigma(t)} dW(t)
\end{equation}
where $\kappa,\bar{\sigma},\gamma$ are constants and the Wiener Process 
satisfies $\forall i = 1,2.....,N$:
\begin{equation}
	d \langle W_{i}, W \rangle _{t} = \rho_{\sigma }dt
\end{equation}

Now, for each asset we take a deterministic barrier, $B_{i}(t)$, or in other 
words a threshold , so that if the asset price falls under that barrier price 
at any time, the firm defaults. We then define a stopping time $\tau_{i}$:

\begin{equation}
	\tau_i = inf\left\lbrace t \geq 0 : S_{i}(t) \leq B_{i}(t) \right\rbrace
\end{equation}

We now define the Portfolio Loss Function $L(t)$ as the number of defaults till 
a given time $t$:
\begin{equation}
	L(t) = \sum_{i =0}^{n} \mathbf{1}_{\lbrace\tau_{i} \leq t \rbrace}
\end{equation}

Since the spreads of CDO tranches are derived from the knowledge of a finite
number of expectations of the form:
\begin{equation}
	\mathbf{E}[(L(T)-K)^{+}]
\end{equation}
where $T$ is the coupon payment date and $K$ is an acceptable number of defaults, 
beyond which we start accumulating losses. So to evaluate such expectation, we 
estimate the probabilities of default. For that purpose, we evaluate 
$\forall k = 0,1,.....,N$
\begin{equation}
	\mathbb{P}(L(T)=k) = \mathbf{p}_{k}(T)
\end{equation}

\subsubsection{Discretization of the Model}
\label{subsubsec:merton_discretization}
For the implementation of our algorithm and for computational efficiency we 
select two time step, $\Delta t = \frac{1}{20}$ which is used to perform the 
selection step, and $\delta t = 10^{-3}$ which will be used in the Euler Step.

The Markov Chain that we simulate is given as ( Note that $X_n$ is $2N + 1$ dimensional):
\begin{equation}
	\label{eq:merton_markov_chain}
	X_{n} = \left( \sigma \left( n \Delta t \right), \left( S_i \left( n \Delta
	t\right) \right)_{1 \leq i \leq N} , \min_{0 \leq m \leq n} \left( \left( S_i
	\left( m \Delta t \right) \right) \right) \right)
\end{equation}

A remarkable and distinctive feature of the Interacting Particle System approach 
is that the evolution dynamics of the underlying process is preserved, that is 
the Markov Chain $X_n$ follows the same evolution dynamics as the continuous Model.

We also take a constant Barrier, $B_i =36$, and define the stopping time $\tau_i$ as :

\begin{equation}
	\tau_i = \min \{ n \geq 0 : S_i(n \Delta t) \leq B_i \}
\end{equation}

Now, we define the potential function such that we assign more weight to portfolios 
with lower values, that is we assign more weight to rare events so the likelihood 
of defaults increases. The potential is a function of $X_p$ and another 
parameter $\alpha > 0$:
\begin{equation}
	\label{eq:merton_potential}
	G_{p}(Y_{p}) = \exp[-\alpha (V(X_p) - V(X_{p-1}))]
\end{equation}

where $V(X_p) = \sum_{i=1}^N \log (min_{0\leq m  \leq p}S_{i}(m \Delta t))$ 
so the potential can be written as:
\begin{equation}
	G_{p}(Y_{p})= \exp\left[ -\alpha \sum_{i=1}^{N}\log\frac{min_{0\leq m  \leq p}
		S_{i}(m \Delta t)}{min_{0\leq m  \leq p - 1}S_{i}(m \Delta t)}\right]
\end{equation}

where $Y_{p} =(X_{0},X_{1},...,X_{p})$ but we only need the last two values 
not the earlier values. Notice that different values of $\alpha$ will give 
different Loss Distributions $\mathbb{P}(L(T) = k)$ for all $k$, because in 
the selection step those portfolios with lower number of defaults are assigned 
a lower weight leading to different number of defaults in each portfolio. The 
choice of the potential function and the parameter $\alpha$ lead to enough 
number of sample paths with large number of defaults even if the number of 
samples is significantly lower than what would be required by a plain Monte 
Carlo simpler. We however follow an idea mentioned in 
\cite{carmona2009importance}, where the best $\alpha$ is selected for each $k$.

\subsubsection{Single Asset Constant Volatility Analysis}
\label{subsubsec:single_asset}
The simplest case of the above model is a case of single asset and no stochastic
volatility. As we know the exact solution for Geometric Brownian Motion, we can 
obtain the hitting time distribution analytically. This can then be used to
compare against estimated values. The solution for the Geometric Brownian Motion
is as follows.
\begin{equation}
	S_t = S_0 \exp \left( \left( r - \frac{\sigma^2}{2} \right) t  + \sigma W_t\right)
\end{equation}
which implies that
\begin{equation}
	\begin{split}
		\mathbb{P}[\tau_B \leq T] = \mathbb{P}[\min_{t\leq T} S_t \leq B] 
		&= \mathbb{P}[\min_{t\leq T} (r - \frac{\sigma^2}{2}) t  + \sigma W_t \leq 
		\log \frac{B}{S_0}] \\
		&= \mathbb{P}[\min_{t\leq T} (r - \frac{\sigma^2}{2}) t  + \sigma \sqrt{t} 
		N(0,1) \leq \log \frac{B}{S_0}]
	\end{split}
\end{equation}

Now, using Girsonov's Theorem, we can get an analytic expression for the 
distribution above:
\begin{equation}
	\mathbb{P}[\tau_B \leq T] = 1 - \left( \Phi(d^+) - \left( \frac{S_0}{B}
	\right)^p \Phi(d^-) \right)
\end{equation}
where :
\begin{equation*}
	d^+ = \frac{ \log \frac{S_0}{B} + (r - \frac{\sigma^2}{2}) t }{\sigma \sqrt{T}}
\end{equation*}
\begin{equation*}
	d^- = \frac{ - \log \frac{S_0}{B} + (r - \frac{\sigma^2}{2}) t }{\sigma \sqrt{T}}
\end{equation*}
\begin{equation*}
	p =  1 - \frac{2r}{\sigma^2}
\end{equation*}
and $\Phi$ is the cumulative distribution function for a standard normal.
